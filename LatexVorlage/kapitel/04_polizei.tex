In der modernen Welt der polizeilichen Ausbildung und Einsatzvorbereitung hat die virtuelle Realität (VR) eine bedeutende Rolle eingenommen. Die Integration von VR-Trainingstechnologien ermöglicht es  Polizeikräften realistische Szenarien zu simulieren und ihre Fähigkeiten in verschiedenen Bereichen zu verbessern. Diese Arbeit widmet sich der Analyse und Evaluation des Einsatzes von VR-Training bei der Polizei, wobei der Fokus auf folgenden Aspekten liegt: dem Umgang mit Stress und Amoksituationen (Abschnitt 3.1), der Forensik und Tatort-Untersuchung (Abschnitt 3.2) sowie der Missionen Planung und Kommunikation (Abschnitt 3.3).
\\
Die zunehmende Komplexität und Vielfalt der Herausforderungen, denen Polizeikräfte gegenüberstehen, erfordert innovative Trainingsmethoden, die über herkömmliche Ansätze hinausgehen. Die Anwendung von VR ermöglicht es den Polizeibehörden, realitätsnahe Szenarien zu schaffen, in denen die Einsatzkräfte ihre Reaktionsfähigkeiten in Stresssituationen trainieren können. Der Abschnitt 3.1 wird daher den Einfluss von VR-Training auf den Umgang mit Stress und Amoksituationen eingehend untersuchen, wobei die Wirksamkeit dieser Technologie bei der Steigerung der Belastbarkeit und Entscheidungsfähigkeit der Polizeibeamten analysiert wird.
\\
Im Abschnitt 3.2 wird der Schwerpunkt auf der forensischen und Tatort-Untersuchung liegen. VR-Technologien bieten ein immersives Umfeld, das es den Ermittlern ermöglicht, Tatorte zu rekonstruieren und forensische Analysen durchzuführen. Diese virtuellen Übungen können die Genauigkeit und Effizienz der Ermittlungen steigern. Die Arbeit wird die Integration von VR in diesen Prozessen eingehend beleuchten und die potenziellen Vorteile für die polizeiliche Arbeit herausstellen.
\\
Schließlich wird der Abschnitt 3.3 sich mit der Missionen Planung und Kommunikation befassen. Die Fähigkeit zur effektiven Planung und Koordination von Einsätzen ist entscheidend für den Erfolg polizeilicher Missionen. Hier wird untersucht, wie VR-Training die Zusammenarbeit und Kommunikation innerhalb von Polizeieinheiten verbessern kann, indem es realistische Simulationen von Einsatzszenarien bereitstellt. 
