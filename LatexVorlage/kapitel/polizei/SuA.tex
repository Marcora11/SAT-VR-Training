Ein großer Schwachpunkt des traditionellen Polizeitrainings ist der Umgang mit Stress und Amoksituationen. Durch normale Schießübungen und Probedurchläufe von Trainingseinheiten mit Dummys ist es unmöglich, eine realitätsgetreue Stresssituation nachzustellen. Wohin entgegen VR-Szenarien eine ähnlich hohe Stresslevel Höhe erreichen können, die sonst nur durch reale Konfliktsituationen erreicht wird. Das führt zur Folge, dass Polizisten echte Amoksituationen besser einschätzen und dementsprechend handeln können. Doch nicht nur die Einschätzung der Situation wird gefördert, sondern auch die Selbsteinschätzung. Durch die realitätsnahe Simulation werden charakteristische Eigenschaften der Polizisten offenbart. Dies dient dazu, falsche Herangehensweisen zu erkennen und zu verbessern, bevor diese in einer realen Situation passieren. Ein weiterer Vorteil des VR-Trainings ist die Wiederholungsmöglichkeit der Trainingslektion. Während bei den traditionellen Übungen Ressourcen wie Munition oder Trainingsdummys begrenzt und schwer wiederverwendbar sind, ist es mit Hilfe einer Virtual Reality Brille und einem Computerprogramm ganz einfach, ganze Trainingseinheiten zu Übungszwecken mehrfach zu wiederholen. \cite{kleygrewe2024virtual}
\\
Durch die Ersparnisse der Dauerkosten aufgrund von Ressourcenbeschaffung ist virtuelles Training trotz hoher Einmalkosten für Brillen und Programme günstiger als herkömmliche Trainingsmethoden. Außerdem werden Kosten gespart, indem veraltete Programme aktualisiert werden können, um dem heutigen Standard zu entsprechen. Deswegen können mehrere Generationen ausgebildet werden, ohne dass die Polizei ständig neue Ressourcen und Materialien zur Verfügung zu stellen muss. \cite{heltne2023cost}
