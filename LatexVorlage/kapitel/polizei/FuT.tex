In der Tatort-Untersuchung ist die VR-Technologie eine immer häufiger auftauchende Alternative. Mit Hilfe einer vorgefertigten 3D-Welt werden Auszubildende in einen virtuellen Tatort versetzt, in dem sie sich frei-bewegen können, dort das Suchen von Spuren und das Bilden von Zusammenhängen zu lernen. Bewegen kann man sich in diesem Raum mit Hilfe von Sensoren, die den Körper tracken, oder durch Benutzung des Controllers. Diese Lernmethode beweist sich als besonders effektiv, da die meisten jungen Erwachsenen heutzutage bereits Erfahrung mit Videospielen haben. Sie empfinden die Übung weniger als Arbeit, da der spielerische Faktor überwiegt. Dieses Spiel ermöglicht es, schneller und einfacher zu lernen, währenddessen Kosten eingespart werden. Der einzige Makel der Anwendung ist die Bewegungskrankheit, die einem ein Übelkeitsgefühl geben kann, da der Körper Unstimmigkeiten zwischen visuell wahrgenommener Bewegung und dem Bewegungssinn aufweist. \cite{mayne2020virtual}
\\
Eine weitere Methode den Auszubildenden die Forensik näherzubringen, ist es, einen Tatort mit einer 360° Kamera einzuscannen und anschließend als Lernübung bereitzustellen. Anders als bei der vorherigen Methode kann hier jeder Schüler ganz einfach mit seinem Smartphone über einen QR-Code den Tatort analysieren. Das hat den Vorteil, dass das Lernen nicht standortabhängig ist und man auch in z.B. der Corona Zeit sein Wissen erweitern kann. \cite{kader2020building}
