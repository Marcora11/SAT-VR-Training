Gute Kommunikation mit den Arbeitskollegen als auch mit Zivilpersonen ist eine Grundvoraussetzung für einen anerkannten Polizisten. Was früher ausschließlich durch Rollenspiele mit anderen Kollegen machbar war, ist heute mithilfe von auf virtuellen Szenarien basierenden Programmen erlernbar. Um hitzige Situationen durch Kommunikation zu entschärfen, werden Polizisten mit Hilfe einer VR-Brille in stressreiche und emotionale Situationen befördert. Das hat zum Vorteil, realitätsnahe Situationen einzustudieren, um bei einem echten Einsatz eigene, durch Stress ausgelöste Fehler zu minimieren. Außerdem erwies sich das Rollenspiel von Person zu Person als makelhaft, da es den Polizisten schwer fiel, den Partner als Rollenspiel Person anzuerkennen, weswegen es oft nötig war, professionelle Schauspieler zu engagieren. Um das empathische Denken weiter zu fördern, müssen Beamte in den VR-Simulationen oftmals auch die Rolle der Zivilperson einnehmen. \cite{kohl2023empathy}
\\
Während die Virtuelle Realität dabei hilft Kommunikation und Empathie zu Zivilpersonen zu erlernen, ist sie außerdem bei der Planung einer Mission und dementsprechend für die Kommunikation der Polizisten untereinander nützlich. 3D Modelle können mit Hilfe der erweiterten Realität (AR), durch geographische Informationen eine Abbildung des realen Einsatzgebietes darstellen. Diese können wiederum durch die VR-Technologie einstudiert und als Trainingseinheit verwendet werden, um das Fehlerrisiko auf ein Minimum zu beschränken.\cite{amorim2013augmented}
