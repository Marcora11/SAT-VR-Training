Die Benutzung der VR-Technologie bringt viele Vorteile mit sich. Sie bietet angehenden und ausgebildeten Polizisten die Chance sich weiterzuentwickeln und in Stresssituationen sich selbst als auch die aktuelle Situation besser einzuschätzen. Es wird ihnen leichter gemacht sich auf Amoksituationen durch VR-Trainingseinheiten vorzubereiten und ein besseres Verständnis zu bekommen was ihre Aufgaben und Vorgehensweisen sind. Hohe Einmalkosten, die durch die Ausstattung der VR-Brillen und Programme aufkommen sind im Gegensatz zu den Dauerkosten des traditionellen Trainings geringer. Durch 360° Scans und virtuell nachgestellte Tatorte fällt es der Forensik leichter, sich weiterzubilden und das teilweise, ohne das Haus verlassen zu müssen. Die Kommunikation wird beim VR-Training nicht vernachlässigt, da stressreiche Situationen nachgespielt und geübt werden können, ohne dass professionelle Schauspieler benötigt werden. Ebenfalls ist die Missionen Planung durch 3D Modelle, die mithilfe erweiterter Realität dargestellt und somit analysiert und studiert werden können, eine wesentliche Erleichterung.
\\
Abschließend kann man sagen, dass das Training in der Virtuellen Realität viele positive Aspekte mit sich bringt und nur wenig bis keine Kritikpunkte aufkommen. Deshalb kann man sich auf die Zukunft Freuen da sich diese Lernmethode immer weiterentwickelt und angewendet wird.
