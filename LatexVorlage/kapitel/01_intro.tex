Das Tutorial enthält verschiedene Sektionen zur Beschreibung grundlegender Funktionen in \LaTeX.\footnote{Vielen Dank an Ihren Kommilitonen Oliver Schneider, der diese Vorlage erstellt hat!}
Zuerst werden die \nameref{sec:basics} in Kapitel~\ref{sec:basics} beschrieben.
Dort sind die \nameref{sec:basics-text}, die \nameref{sec:basics-lists}
und das \nameref{sec:basics-contents} beschrieben.
Darauf folgend ist die Verwendung und Erstellung von \nameref{sec:BilderTabellenListings} in Kapitel~\ref{sec:BilderTabellenListings} erklärt.
Verwaltung und richtiges Zitieren in \LaTeX~ist im Kapitel~\ref{sec:bibliography} zu finden.
Die Verwendung von einem Glossar und Akronym-Verzeichnis ist im Kapitel~\ref{sec:glossary} enthalten.
Wichtige Quellen dieser Arbeit sind~\cite{Dermeval2015},\cite{pohl2016requirements}.

\LaTeX~kann auf nahezu allen Plattformen (Mac,Linux, Windows und Online) installiert werden.
Hierfür kann der Link in der Fußzeile aufgerufen werden.

Für die effiziente Bearbeitung Ihrer Arbeit empfehlen wir Texmaker\footnote{\url{https://www.xm1math.net/texmaker/}}.
Dafür brauchen Sie eine unterliegende LaTeX Installation, z.B.~die Tex-Live Installation\footnote{\url{http://tug.org/texlive/}}.
Meist reicht die Standardinstallation.
Wenn Sie die Tex-Dateien manuell kompilieren wollen, müssen Sie in einer Kommandozeile folgendes tun:
\begin{enumerate}
    \item Terminal öffnen und mit \lstinline|cd| in den Hauptordner des Projekts navigieren
    \item \lstinline|pdflatex tutorial.tex| eingeben und mit Eingabetaste bestätigen.
    \item \lstinline|makeglossaries tutorial| erstellt die Akronyme und das Glossar.
    \item Die Befehle \lstinline|bibtex tutorial| und  \lstinline|biber tutorial| Befehl erstellen das Literaturverzeichnis.
    \item Erneut \lstinline|pdflatex tutorial.tex| ausführen, um Akronyme, Glossar und das Literaturverzeichnis einzufügen.
\end{enumerate}


In \LaTeX kann man ein Wort mit \lstinline|\textbf{wort}| \textbf{fett} und mit
\lstinline|\textit{wort}| \textit{kursiv} schreiben.\\
\par
Ein Sprung in eine neue Zeile in einem \LaTeX Dokument wird nicht mit
der Eingabetaste, sondern mit den Zeichen \lstinline|\\| erreicht.\\
\\
\textbf{Beispiel:}\\
Nach diesem Satz wird eine neue Zeile begonnen.\\
Das ist die neue Zeile.\\
\par
Eine neuer Absatz ist mit dem Befehl \lstinline|\par| möglich.\\
\\
\textbf{Beispiel:}\\
Nach diesem Satz wird ein neuer Absatz entstehen.\\
\par
Dieser Satz steht in einem neuen Absatz.\\
\par
Nach dem Befehl \lstinline|\newpage| wird auf der Text auf der folgenden Seite fortgesetzt.\\
\\
\textbf{Beispiel:}\\
Der nächste Satz wird auf einer neuen Seite stehen.\\
\newpage
Dieser Satz steht auf einer neuen Seite.
