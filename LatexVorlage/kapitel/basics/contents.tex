Das Inhaltsverzeichnis wird in der \path{tutorial.tex}
definiert.
Die Hauptkapitel sind durch \lstinline|\chapter{Name des Kapitels}| eingeleitet.
Unterkapitel werden in erste Stufe mit \lstinline|\section{Name des Unterkapitels}| 
oder in zweiter Stufe mit \lstinline|\subsection{|\\\lstinline|Name des Unter-Unterkapitels}| erzeugt.
Um Verlinkungen und Referenzen auf die Kapitel und Unterkapitel erstellen zu können,
wird den Kapiteln ein label mit \lstinline|\label{sec:Kapitelname}| hinzugefügt.
Mit \lstinline|~\ref{sec:Kapitelname}| kann die Nummer des Kapitels oder Unterkapitels
in den Text eingefügt werden und mit \lstinline|\nameref|\\\lstinline|{sec:kapitel-1}| 
der Name des Kapitels.\\
\textbf{Beispiel:}
Das folgende Kapitel ist das Kapitel~\ref{sec:basics-subsection}.\\
Das folgende Kapitel ist das \nameref{sec:basics-subsection}.\\

