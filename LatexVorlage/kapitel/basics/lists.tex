Die beschreibende Liste in \LaTeX~wird mit \lstinline|\begin{description}|
eingeleitet. Die einzelnen Punkte werden durch \lstinline|\item[Wort] Beschreibung| erstellt.
Abgeschlossen wird die Liste mit \lstinline|\end{description}|.\\
\\
\begin{description}
    \item[LaTeX] wird zum wissenschaftlichen Schreiben verwendet
    \item[Java] ist eine Programmiersprache
    \item[\ldots]
\end{description}
Die nummerierte Liste in \LaTeX~ist wird mit \lstinline|\begin{enumerate}| begonnen.
Die einzelnen Punkte werden durch \lstinline|\item | erstellt.
Abgeschlossen wird die Liste mit \lstinline|\end{description}|.\\
\\
\begin{enumerate}
    \item CPU
    \item RAM
    \item \ldots
\end{enumerate}
Die einfachste Art der Liste in \LaTeX~ist die Punktliste,
die aber \textbf{nicht in wissenschaftlichen Arbeiten eingesetzt} wird.
Diese Liste wird mit \lstinline|\begin{itemize}| begonnen.
Die einzelnen Punkte werden, wie bei der nummerierten Liste, durch \lstinline|\item Punkt1| erstellt.
Abgeschlossen wird die Liste mit \lstinline|\end{itemize}|.\\
\\
\begin{itemize}
    \item CPU
    \item RAM
    \item \ldots
\end{itemize}
