{\small
\textbf{Abstract:}\\
Virtual Reality (VR) gilt als vielversprechende Technologie, die heutzutage nicht mehr
wegzudenken ist. Ein Grund dafür ist, dass es noch nie so einfach war, komplexe Inhalte mit
einem hohen Potenzial für Interaktivität zu vermitteln. Die Vorteile virtueller Trainings sind
vor allem die Schulungen bei gefährlicher Arbeit oder wenn diese kosten- und zeitaufwendig
sind. Deshalb wird in der in der Raumfahrt vermehrt das Training mit VR eingesetzt. Für die
geplanten neuen Missionen der NASA ist die Entwicklung neuer Trainings und der
zugehörigen VR- Technologien erforderlich. Dabei kann auf etliche vorhandene
Entwicklungen zurückgegriffen werden. Diese Literaturarbeit befasst sich in Kapitel 2 mit den Systemen
Simplified Aid for EVA Rescue (SAFER) SAFER und Charlotte einem Mass Handling
System.
Bei Kapitel 3 behandeln wir das VR-Training bei Fahrzeugen. Darunter wird das VR-Training bei Piloten, Bodenfahrzeugen und der Automanufaktur behandelt. Dabei gehen wir einzeln auf wie es funktioniert, die Effektivität und den Ausblick ein.
Das Verständnis dieser Systeme kann helfen, zukünftige Systeme zu designen und
zu entwickeln
}