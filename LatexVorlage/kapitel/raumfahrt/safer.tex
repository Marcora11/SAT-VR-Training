Bei SAFER (Simplified Aid for EVA Rescue) handelt es sich um System, dass zur Selbstrettung verwendet wird. \cite{moore201021st}
Es ist während EVAs am Raumanzug befestigt und wird eingesetz, wenn ein Atronaut unabsichtlich von der ISS getrennt wird. \cite{miralles2013onboard}
SAFER besteht aus einem Triebwerksrucksack mit gasförmigem Stickstoff. \cite{moore201021st}
Es wird deshalb auch als "Jetpack" bezeichnet. \cite{garcia2020training}.
Der letzte Einsatz von SAFER liegt 30 Jahre zurück, trotzdem ist SAFER weiterhin für jeden EVA notwendig.\cite{garcia2020training}
\\
Die Simulation wird über ein VR Dead Mountet Display (HMD) dargestellt. \cite{garcia2020training}
Garcia et al. führen aus, dass beim erste VR Headset der Systems 2012 ein Laptop auf den Kopf des Astronauten geschnallt wurde.
Dadurch konnte die VR Technologie auch auf der ISS das erste Mal zum trainieren benutzt werden. \cite{garcia2020training}
2020 wurden dann laut Garcia et al. die Vive Pro HMDs eingesetzt. Dazu kommen noch Hand und Körpertracking sowie eine Handgestenerkennung. \cite{garcia2020training}
\\
Die SAFER Simulation beinhaltet die Physik-, Dynamik- und Sensordaten sowie Modelle für die Flugeigenschaften, Energie und Triebwerk. \cite{garcia2020training}
Simuliert werden kann eine Überprüfung des SAFER Systems, welche direkt vor einem EVA durchgeführt wird.
Der Ausbilder kann auf das Interface der Simulation zugreifen und kann Fehlermeldungen hervorrufen. Die Astronauten trainieren so, wie sie bei Fehlern reagieren müssen. \cite{garcia2020training}
Garcia et al. beschreiben noch eine zweite Einsatzmöglichkeit. Diese ermöglicht es, das SAFER system in VR zu fliegen. \cite{garcia2020training}
Laut Moore et al. wird beim Training ein Austronaut in der virtuellen Welt von der ISS getrennt und dieser muss sich selbst durch den Einsatz von SAFER retten. \cite{moore201021st}
Im Trainingsszenario taumelt der Astronaut 30 Sekunden von der ISS weg. Danach muss erfolgreich zurück zur ISS fliegen. Astronauten proben den Flug von SAFER viele Male mit unterschiedlichen Konfigurationen. Am Ende muss ein Prüfungsflug bestanden werden.
Auch an Bord der ISS hat der Austronaut dann nochmals die Möglichkeit zu trainieren. \cite{garcia2020training}
Ein Astronaut muss diese Technik der Selbstrettung gut beherrschen, da ihm sonst der Treibstoff ausgehen kann. \cite{moore201021st}
