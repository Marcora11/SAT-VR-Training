Moore et al. beschreiben, dass die VR Trainingsstation für die akkurate Simulation des Umgangs mit schweren Bauteilen in der Schwerelosigkeit benutzt wird.\cite{moore201021st}
Charlotte besteht aus einem oder zwei Robotern. Dabei können zwei Astronauten gleichzeitig am selben Bauteil üben.\cite{miralles2013onboard}
Garcia et al. stellen in ihrer Arbeit die Trainingsstationen dar.
Die Trainingsumgebung ist so aufgebaut, dass zwei Astronauten sich Rücken an Rücken sitzen. Sie befinden gleichzeitig sich in einer verbundenen virtuellen Umgebung, bedienen aber zwei physische Charlotte Roboter.
Beide Astronauten tragen Vive Pro HMD, Handschuhe mit je einem Vive tracking Vorrichtung und einem Tracker für den Torso.
Die HMDs werden mit Windows PCs betrieben, die mit einer Nvidia 1080Ti Grphikkarte ausgestattet sind.
Außerdem läuft DOUG auf einem Linux Server Prozess. \cite{garcia2020training}
Mit Charlotte können Astronauten mit simulierten Bauteilen üben, die unterschiedlich schwer, groß und gewichtet sind. \cite{miralles2013onboard}
Charlotte wurde 1997 in das Training der Astronauten integriert und hat den Vorteil gegenüber früheren Traingsmethoden, das das Erscheinungsbild der Übungsmassen leicht verändert werden konnte. \cite{garcia2020training}
\\
Garcia et al. beschreiben , dass die auf der Interfaceplatte, Sensoren die vom Astronaut gewirkten Kräfte und Drehungen registriert.
Die Interfaceplatte ist die Schnittstelle zwischen Astronaut und simuliertem Bauteil, also die Griffe, die der Astronaut in der Hand hält.
Die DOUG Grafik wird dann auf dem HDM des Astronauten und dem Bildschirm des Ausbilders entsprechend geupdated. \cite{garcia2020training}
\\
Charlotte kann einfach verändert werden und kann dann andere pyhsikalische Bauteile simulieren.
Dabei müssen keine echten Gegenstände modelliert werden. \cite{garcia2020training}
Die Astronauten können solange mit Charlotte trainieren, bis sie sich mit dem Umgang von großen Massen in der Schwerelosigkeit wohl fühlen.\cite{garcia2020training}
