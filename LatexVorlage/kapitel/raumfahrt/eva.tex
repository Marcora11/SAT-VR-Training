Das Virtual Reality Labroratory ist eine der Trainingsstätten für das EVA Training am NASA Johnson Space Center(JSC). \cite{moore201021st}
Das VR LAb ist der einzige Ort, an dem Astronauten überall auf der ISS trainieren können, da es wegen der Größe der ISS keine echten Modelle gibt.
Die Mehrheit der durchgeführten VR Trainings beschäftigen sich mit EVAs.
Das VR Training wird hier zusätzlich zu anderen Methoden eingesetzt. \cite{osterlund2012virtual}
Osterlund et al. beschreiben weitere Vorteile des VR Trainings.
Durch die relative Position des Astronauten zur ISS und den zur Mission gehörenden Teilen können die Position des Piloten und des Roboterarmes korrigiert werden.
Dabei findet das Training in einer sicheren Umgebung statt. \cite{osterlund2012virtual}
\\
Moore et al. beschreiben, dass dort komplizierte EVAs am besten visualisiert und analysiert werden können. Außerdem kann das VR Training für Robotermanöver nur dort akkurat durchgeführt werden.
Die Austronauten studieren dort auch ihre Kommunikation und die Zeitliche Abfolge genau ein. \cite{moore201021st}
Miralles schreibt in ihrer Arbeit, dass Astronauten für verschiedene Arbeitsplätze auf der ISS üben können und sich damit vertraut machen können, welche Wege sie für EVAs nehmen sollten. \cite{miralles2013onboard}
Die eingesetzte Software heißt Dynamic Onboard Ubiquitious Graphics (DOUG). Sie kann auch an Bord der ISS genutzt werden, um bevorstehende EVAs zu trainieren. Dies macht die Astronauten selbstbewusster neue Techniken und Schritte anzuwenden. \cite{osterlund2012virtual}
Dies ist notwendig, da viele EVAs Reperatur Aufgaben geworden sind, die nicht vorher auf der Erde trainiert wurden.\cite{miralles2013onboard}
Osterlund et al. beschreiben DOUG als eine 3D Animation der ISS in der neuesten Konfiguration.
\\
Sie wird nicht nur für das Training, sondern auch in der Planung und beim Nachvollziehen der Arbeitsschritte eingesetzt.
Es handelt sich dabei um einen Prototypen, der eine billige Lösung darstellt.  \cite{osterlund2012virtual}
Das liegt daran, dass das System schnell angepasst weren kann und so eine Vielzahl von Szenarien evaluiert werden können.
DOUG kann für hochrealistische Trainingsszenarien eingesetzt werden. \cite{miralles2013onboard}
Astronauten können sich mit dem Arbeitplatz für das Space Station Remote Manipulator System (SSRMS) vertraut machen. Damit können Bauteile und Astronauten außerhalb der ISS bewegt werden.\cite{garcia2020training}
Miralles bescheibt, dass im VR Lab zusätzlich realistische Lichtverhältnisse simuliert werden.
Außerdem können Szenarien mit dem Einsatz des Roboterarmmes und zwei weiteren Astronauten trainiert werden. \cite{miralles2013onboard}
\\
Das Training für EVAs findet normalerweise Monate vor dem Flug zur ISS statt. Spezifische Aufgaben werden auf der Erde geplant und dann eine Gruppe von Astronauten speziell darauf trainiert. \cite{garcia2020training}