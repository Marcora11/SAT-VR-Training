
Es wird eine Simulation eines Operationssaals erstellt in der Ärzte den Operationssaal näher kennenlernen können und sich mit den verschiedenen Werkzeuge vertraut machen können. Es können verschiedene schwierige Operationen simuliert werden. Es gibt bei Operationen verschiedene Arten wie Virtual Reality zum Einsatz kommt.\\
Operationssimulation:\\
Ein virtueller Operationsraum in dem frei geübt werden kann. Die Werkzeuge sind alle virtuell und durch einen Controller steuerbar. In der Simulation können verschiedene Operationen mit unbegrenzten Ressourcen getestet werden. Ein Fehlversuch der Operation führt zu einem Neustart der Simulation, was eine endlose Testmöglichkeit bietet\\
3D-Body- Mapping:\\
Mit Hilfe von Augmented Reality lassen sich verschiedene Organe auf einem Gerät des zu behandelnden Patienten anzeigen und der Arzt kann die Lage der einzelnen Organe besser einschätzen.\\   \cite{mehlitz1998virtual}
Operationsroboter:\\
Eine echte Operation mit Hilfe eines Operationsroboters welcher über eine VR-Brille gesteuert wird.

Vor- und Nachteile\\
Operationssimulation:\\
Vorteile:\\
-Schwere Operationen
können ohne Konsequenzen gelehrt und getestet werden
-Eine vollständige Simulation wirkt realer
Nachteile:\\
Kann reale Szenarien nicht perfekt widerspiegeln
3D-Body- Mapping:
Vorteile:\\
-Präzisere Arbeit
-Besserer Blick auf die Organe des Patienten
Nachteile:\\
-Kann teils zu ungenau sein
Operationsroboter:
Vorteile:\\
-Arbeitet genauer als ein Mensch
Nachteile:\\
-Schulung zur Benutzung nötig
-Kostenaufwendig

Fazit\\
Ein VR-Training bei Operationen ist sehr wichtig, da in der Medizinbranche direkt mit dem Patient gearbeitet wird und bei Fehlern fatale Folgen entstehen können. Das VR-Training bietet Ärzten somit eine perfekte Vorbereitung für verschiedene Operationen ohne jegliche Konsequenzen mit sich zu bringen.


\footnote{\url{https://vr-dynamix.com/virtual-reality-medizin/}}

