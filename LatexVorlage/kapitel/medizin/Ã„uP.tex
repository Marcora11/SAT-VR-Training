
In der Medizin gibt es mittlerweile auch verschiedene Trainingseinheiten für das Beseitigen von Ängsten oder Phobien. Dabei werden verschiedene Simulationen durchgeführt in denen die Patienten ihren Ängsten ausgesetzt sind. Dies wird auch Expositionstherapie genannt.\\

Methoden\\
Ängste und Phobien:\\
Patienten werden in einer Simulation ihren Ängsten ausgesetzt. Haben keinen Einfluss auf das Geschehen also nur visuell\\

Schmerztherapien:\\
Patienten werden in einer Simulation von dem Schmerz abgelenkt. Die Therapie fokussiert sich darauf das Gehirn zu entspannen oder die Patienten vor einem Trauma zu bewahren\\

Vor- und Nachteile\\

Ängste und Phobien:\\
Vorteile:\\
-Es sind keine Medikamente notwendig\\
-Personal wird entlastet\\
Nachteile:\\
-spiegelt nicht Realität wieder\\

Schmerztherapie:\\
Vorteile:\\
-Abhängigkeit von Medikamenten nimmt ab, da man einer Therapie ausgesetzt ist und die Patienten langsam von den Schmerzmitteln wegkommen können.\\
-Ablenkung der Patienten macht bestimmte Aufgaben leichter, wie zum Beispiel das Spritzen bei Kindern, wenn diese durch ein VR-Programm abgelenkt werden.\\

Fazit\\
Ein solches Training ist besonders sinnvoll, da es die Patienten direkt mit ihren Ängsten und Phobien in einem gespielten Szenario konfrontiert und sie sich langsam gegenüber diesen abhärten. Bei einer Schmerztherapie kann man Menschen vor einem Trauma bewahren und die Patienten von ihren Schmerzen ablenken als auch zu mindern. 





