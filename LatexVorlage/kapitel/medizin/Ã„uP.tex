
In der Medizin gibt es mittlerweile auch verschiedene Trainingseinheiten für das Beseitigen von Ängsten oder Phobien. Dabei werden verschiedene Simulationen durchgeführt in denen die Patienten ihren Ängsten ausgesetzt sind. Dies wird auch Expositionstherapie genannt.\\

\subsection{Methoden}
\subsection{Ängste und Phobien:}
Patienten werden in einer Simulation ihren Ängsten ausgesetzt. Haben keinen Einfluss auf das Geschehen also nur visuell. Dieses Verfahren hat den Vorteil, dass keine Medikamente nötig sind, da der Patient einer virtuellen Stresssituation ausgesetzt ist. Auch wichtig zu erähnen ist, dass vor allem Personal im Therapiebereich entlastet wird und die Arbeit mit Hilfe von VR-Brillen einfacher ist. Ein wichtiger Nachteil der Methode ist, dass solch ein Szenario nicht die Realität widerspiegelt und Fortschritte manchmal nur schwer zu erreichen sind. Vor allem liegt dies daran, dass das Verfahren nur visuell ist.\\

\subsection{Schmerztherapien:}
Patienten werden in einer Simulation von dem Schmerz abgelenkt. Die Therapie fokussiert sich darauf das Gehirn zu entspannen oder die Patienten vor einem Trauma zu bewahren. Vorteile solch einer Therapie sind vor allem die Reduzierung von Schmerzmitteln und die dazu gehörige Abhängigkeit sollte stark abnehmen. Eine solche Therapie versucht den Patienten komplett von den Schmerzmittel zu befreien. Bei einem gewissen Trauma sind gewisse Schmerztherapien besonderns hilfreich, da eine Simulation den Patienten ein schönes Bild widergibt um diese von ihrem Trauma abzulenken. Dassellbe könnte auch bei Kindern angewendet werden, um diese bei einer Impfung ruhigzustellen.\\
<<<<<<< HEAD

\subsection{Vor- und Nachteile}
\subsection{Ängste und Phobien:}
\subsection{Vorteile:}
-Es sind keine Medikamente notwendig\\
-Personal wird entlastet\\
\subsection{Nachteile:}
-spiegelt nicht Realität wieder\\

\subsection{Schmerztherapie:}
\subsection{Vorteile:}
-Abhängigkeit von Medikamenten nimmt ab, da man einer Therapie ausgesetzt ist und die Patienten langsam von den Schmerzmitteln wegkommen können.\\
-Ablenkung der Patienten macht bestimmte Aufgaben leichter, wie zum Beispiel das Spritzen bei Kindern, wenn diese durch ein VR-Programm abgelenkt werden.\\
=======
>>>>>>> a886105 (Update ÄuP.tex)

\subsection{Fazit}
Ein solches Training ist besonders sinnvoll, da es die Patienten direkt mit ihren Ängsten und Phobien in einem gespielten Szenario konfrontiert und sie sich langsam gegenüber diesen abhärten. Bei einer Schmerztherapie kann man Menschen vor einem Trauma bewahren und die Patienten von ihren Schmerzen ablenken als auch zu mindern.





