
Bei diesem Training werden Patienten vor allem bei Lähmungen verschiedene Aufgaben in einer Simulation gegeben und diese schrittweise erfüllt. Die Patienten bekommen eine Extralerneinheit zusätzlich zum eigentlichen Rehabilitationsprogramm\\

Idee\\
Rehabilitation ist anstrengend und meist ist kaum Fortschritt spürbar durch ein zusätzliches VR-Programm wäre es deutlicher einfacher und Fortschritt wird den Patienten auch sichtbar gemacht.Dies sollte die Patienten motivieren ständig am Programm zu bleiben und sich gesundheitlich verbessern. Durch Spiegelung soll der Patient langsam Kontrolle über eigenen Körper wiedererlangen, in dem der Patient durch das VR-Programm sieht, wie dieser eine virtuelle Hand steuern kann. Dies würde im Gehirn dann Nervenzellen aktivieren, welche den echten Arm Reize senden und diesen langsam wieder bewegen lassen.\\

Fazit\\
Ein VR-Training für Rehabilitation erweist sich besonders als hilfreich, da Patienten ihren eigenen Fortschritt erkennen können. Der eigene Fortschritt könnte diese dann motivieren und positive Effekt auslösen was zu einer schnelleren Genesung führen könnte. Es sollte aber nicht dass Programm ersetzen sondern es jeglich ergänzen


