4.3 VR-Training für Reha\\

Bei diesem Training werden Patienten vor allem bei Lähmungen verschiedene Aufgaben in einer Simulation gegeben und diese schrittweise erfüllt. Die Patienten bekommen eine Extralerneinheit zusätzlich zum eigentlichen Rehabilitationsprogramm\\

Idee\\
-Reha ist anstrengend und meist ist kaum Fortschritt spürbar 
durch ein VR-Training ist Fortschritt messbar und somit auch sichtbarer\\
-soll motivieren durch Extraprogramm\\
-Durch Spiegelung soll der Patient langsam Kontrolle über eigenen Körper wiedererlangen\\

Fazit\\
Ein VR-Training für Rehabilitation erweist sich besonders als hilfreich, da Patienten ihren eigenen Fortschritt erkennen können. Der eigene Fortschritt könnte diese dann motivieren und positive Effekt auslösen was zu einer schnelleren Genesung führen könnte. Es sollte aber nicht dass Programm ersetzen sondern es jeglich ergänzen


