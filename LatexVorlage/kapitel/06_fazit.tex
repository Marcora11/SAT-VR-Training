\section{Zusammenfassung der Ergebnisse}
In der Raumfahrt ist das VR Training zu einem wichtigen Teil der Ausbildung der Astronauten geworden.
Jeder Astronaut, der zur ISS fliegt, muss VR Training an lebenswichtigen Systemen absolvieren. Daneben wird VR Technologie auch für Entwicklung von Arbeitsabläufen für EVAs genutzt.
Die VR Technologie hat den Vorteil, dass sie kostengünstig ist und das sie schnell angepasst werden kann. Sie wird sowohl im VRLAB am NASA Johnson Space Center, als auch an Bord der ISS erforgreich eingesetzt.

Beim Training der Benutzung der Fahrzeuge ist VR ein fortschrittlicher Bestandteil.
Bei Autos ist es schon ziemlich weit entwickelt und fast an der Grenze. Bei Piloten ist es ziemlich weit aber hat Luft nach oben und wird wahrscheinlich in Zukunft immer häufiger benutzt und verbessert.
Bei der Automanufaktur wird es noch ziemlich wenig benutzt, jedoch wird dadurch hier der größte Anstieg der Benutzung erwartet und wir persönlich vermuten das es auch hier und Zukunft fast immer benutzt wird.
Bei allen Bereichen wird es in Kombinationen benutzt um beste Ergebnisse zu bekommen.

VR-Training ist für den Lernprozess von Ärzten essenziell und bietet neue Lernmethoden, welche zuvor nicht möglich gewesen sind. Extremoperationen können durchgeführt und geschult werden, ohne dass Menschenleben riskiert werden und Ärzte können besser auf Operationen vorbereitet werden.
\section{Weitere Arbeiten}
Welche neue Ideen haben sich ergeben?
Was müsste weiter untersucht werden?
Welche weiteren Bachelor- oder Masterarbeiten sind in dem Themenfeld nun interessant geworden?
