\usepackage[utf8]{inputenc}
\usepackage[T1]{fontenc}

% Deutsches Sprachpaket für Babel auswählen
\usepackage[german] {babel}

% Font Änderungen
\usepackage{lmodern}
\usepackage[scaled]{helvet}
\renewcommand\familydefault{\sfdefault} 

% Integration von PDF Seiten. Wird zum Einfügen der eidesstattlichen Erklärung der Thesis verwendet.
\usepackage{pdfpages}

% Lipsum Fließtext generieren mit \lipsum 
\usepackage{lipsum}
\usepackage{cprotect}
\usepackage{csquotes}

% Kopf- und Fußzeilen der Seiten anpassen 
\usepackage{fancyhdr}

% Ersetzt \hline in Tabellen mit \toprule, \midrule und \bottomrule
\usepackage{booktabs}

% Unter anderem für mathematische Formeln
\usepackage{amsmath}

% Enthalten mathematische Symbole
\usepackage{amsfonts}
\usepackage{kpfonts}
\usepackage{amssymb}

% Generiert graphische Elemente, wie beispielsweise Diagramme
\usepackage{tikz}

% Bilder in Latex einbinden mit \includegraphics
\usepackage{graphicx}

% Den Pfad für alle Bilder relativ zur header.tex-Datei setzen.
\graphicspath{ {images/} }

% Tabellen automatisch an die Textbreite anpassen
\usepackage{tabularx}

% Mehrere Reihen in einer Tabelle zusammenschließen
\usepackage{multirow}

% Mehr Farben für LaTeX
\usepackage{xcolor}
\usepackage{color}
\definecolor{dkgreen}{rgb}{0,0.6,0}
\definecolor{dkblue}{rgb}{0,0,0.2}
\definecolor{gray}{rgb}{0.5,0.5,0.5}
\definecolor{mauve}{rgb}{0.58,0,0.82}
\definecolor{darkerred}{rgb}{0.2,0,0}

% Paket, um Links im Dokument zu erzeugen
\PassOptionsToPackage{hyphens}{url}\usepackage{hyperref}

% Farben und Art der Links anpassen
\hypersetup{
    colorlinks = true,
    linkcolor=black,
    filecolor=black,      
    urlcolor=black,
    citecolor=black
}

% Enumerationsliste mit römischen Zeichen
\renewcommand{\theenumi}{\roman{enumi}}
\usepackage{footnote}
\makesavenoteenv{figure}
\usepackage{epigraph}
\usepackage{listings}

% Umlaute in Listings zulassen
\lstset{literate=% Allow for German characters in lstlistings.
{Ö}{{\"O}}1
{Ä}{{\"A}}1
{Ü}{{\"U}}1
{ß}{{\ss}}2
{ü}{{\"u}}1
{ä}{{\"a}}1
{ö}{{\"o}}1
}

% Farben in Lstlistings verwenden
\lstset{frame=tb,
    language=Java,
    aboveskip=3mm,
    belowskip=3mm,
    showstringspaces=false,
    columns=flexible,
    basicstyle={\small\ttfamily},
    numbers=none,
    numberstyle=\tiny\color{black},
    keywordstyle=\color{black},
    commentstyle=\color{black},
    stringstyle=\color{black},
    breaklines=true,
    breakatwhitespace=true,
    tabsize=3
}

% Paket für Akronyme und Glossar
\usepackage[acronym,nonumberlist,order=letter,nopostdot,toc,numberedsection=autolabel,section=chapter]{glossaries}
%,style=super
\renewcommand*{\glspostdescription}{}
\setacronymstyle{long-short}

% Ändern der Farbe für Verlinkungen auf die Akronyme und in das Glossar auf ein dunkles Rot.
\renewcommand*{\glstextformat}[1]{\textcolor{black}{#1}}
\makenoidxglossaries{}

\usepackage[doublespacing]{setspace}
\usepackage{background}
\usepackage{lastpage}

% Mit dem Command \subautor kann jedem einzelnen Kapitel ein einzelner Autor hinzugefügt werden
\newcommand{\subautor}[1]{\begin{flushright}
    {\small\textit{#1}}    
\end{flushright}}

% Notwendig, um die Verlinkung in das Inhaltsverzeichnis über die Fußzeile zu erstellen.
\backgroundsetup{contents={}}

% Normaler Stil für Seiten innerhalb der Arbeit
\pagestyle{empty}
\renewcommand{\headrulewidth}{0pt}% removes header line
\lhead{\leftmark}
\rhead{}
\cfoot{\thepage}
\lfoot{\hyperlink{contents}{\small{Inhaltsverzeichnis}}}% links the TOC at the center of the page footer

% Am Ende der Arbeit werden die Kapitelreferenzen in der Kopfzeile entfernt.
\renewcommand{\headrulewidth}{0pt}% removes header line
\lhead{}
\rhead{}
\cfoot{\thepage}
\lfoot{}

% Am Anfang des Dokuments werden alle Kopf- und Fußzeilen entfernt.
\AtBeginDocument{\addtocontents{toc}{\protect\thispagestyle{empty}}}
